\documentclass[12pt]{article}
\usepackage{graphicx}
\usepackage{amsmath,amssymb}
\usepackage{hyperref}
\usepackage{fontspec}
\usepackage{setspace}
\usepackage[margin=1in]{geometry}
\usepackage[table,xcdraw]{xcolor}
\usepackage[textfont={rm,it}]{caption}
\usepackage{subcaption}
\usepackage{subfloat}
\usepackage{multirow}
\usepackage{multicol}
\usepackage[american]{babel}
\usepackage{csquotes}

\begin{document}
\title{Annealed Pressure of Ising Model on Sparse Bipartite Random Regular Graphs}
\author{Yizhong Hu}
\maketitle

\begin{abstract}
    The Ising model, initially introduced in the early 1900s to study magnetization in statistical mechanics, has since become a versatile tool applied in various fields like social sciences, genetics, and combinatorics. It assigns probabilities to configurations of 1's and -1's, representing dipole spins, on a graph or network. In the ferromagnetic model, neighboring 1's are favored, while in the antiferromagnetic model, opposite signs are preferred. Over time, the Ising model has been extended from the traditional d-dimensional lattice to more general network or graph structures.

    In this study, we explore the Ising model on random graph geometries as the number of particles approaches infinity. Our focus is on a quantity called the 'annealed pressure,' which helps us understand key aspects of the model, such as the limiting magnetization. To do this, we employ a large deviation approximation and express the annealed pressure as an optimization problem. Through numerical and analytical techniques, we investigate the Ising model on random regular graphs and random regular bipartite graphs. In particular, we establish a link with a quantity referred to as the 'Bethe prediction' through a certain cavity equation, proving that the Bethe prediction is equivalent to the annealed pressure in specific statistical physics models. We also demonstrate the method's applicability to other statistical physics models on random graphs, such as the Hardcore model. Our research contributes to a deeper understanding of statistical physics and optimization theory, particularly in the context of complex systems on random graph structures.
\end{abstract}

\end{document}